\documentclass[12pt]{article}
\usepackage{braket}
\usepackage{physics}
\usepackage{graphicx}
\usepackage{times}
\usepackage[export]{adjustbox}
\usepackage{listings}
\usepackage{mathcomp}
\usepackage{hyperref}
\usepackage{bm,amsmath}
\usepackage{amssymb}
\usepackage{float}
\usepackage{indentfirst}
\usepackage{bigints}
\usepackage{listings}
\usepackage{color}
\hypersetup{
colorlinks=true,
linkcolor=blue,
filecolor=magenta,
urlcolor=cyan,
pdftitle={Overleaf Example},
pdfpagemode=FullScreen,
}
\definecolor{dkgreen}{rgb}{0,0.6,0}
\definecolor{gray}{rgb}{0.5,0.5,0.5}
\definecolor{mauve}{rgb}{0.58,0,0.82}
\lstset{frame=tb,
language=Python,
aboveskip=3mm,
belowskip=3mm,
stepnumber = 1,
showstringspaces=false,
columns=flexible,
basicstyle={\small\ttfamily},
numbers=left,
numberstyle=\color{gray},
keywordstyle=\color{blue},
commentstyle=\color{dkgreen},
stringstyle=\color{mauve},
breaklines=true,
breakatwhitespace=true,
tabsize=3
}
\numberwithin{equation}{section}

\title{Note For Noether's Theorem}
\author{Ting-Kai Hsu}
\date{\today}

\begin{document}
\maketitle
\tableofcontents
\section{Noether Theorem in Classical Mechanics}
\subsection{Lagrangian Formalism}
In Lagrangian formalism, we've defined the action by lagrangian,
\begin{equation}
    S = \int_{t_1}^{t_2}{L\left[x(t), \dot{x}(t)\right]\,dt}
\end{equation}\label{1.1}
\indent If action follows the \textit{least action principle}, we would have the lagrangian be associated with the correct \textit{equation of motion} of the system given by eq(\hyperref[1.1]{1.1}).
\begin{equation}
    \frac{\partial L}{\partial x_i} - \frac{d}{dt}\frac{\partial L}{\partial \dot{x}_i} = 0
\end{equation}\label{1.2}
\indent Noether's theorem states that if given system has continuous symmetry, then there would be a conserved quantity associated with this symmetry.
\\\indent What is continuous symmetry? 
One that leaves the action invariant even when the dynamical equations (equations of motion) are \textit{not} satisfied, we call it \textbf{infinitesimal symmetry transformation}. 
We denote the infinitesimal symmetry transformation as $\delta_{\text{S}}$ to separate it from the variation $\delta$.
\\\indent We're interested in the case when the action doesn't change under a symmetry, and this implies there is special property behind the dynamics of the system.
% Consider the general infinitesimal transformation, 
% \begin{equation}
%     \begin{split}
%         t\rightarrow t' = t+\epsilon\frac{dt}{d\epsilon}+\cdots\\
%         x(t)\rightarrow x'(t')=\mathcal{F}(x(t)) = x(t)+\frac{d\mathcal{F}}{d\epsilon}\epsilon
%     \end{split}
% \end{equation}\label{1.3}
% Where $\epsilon$ is infinitesimal parameter.\\\indent
% Define the variation corresponding to the symmetry, and it should be expressed as infinitesimal transformation at the same point (or time, or generally coordinates).
% \begin{equation}
%     \delta_{\text{S}}x = x'(t) - x(t)
% \end{equation}
% \indent Let's find out how would the new action look,
% \begin{equation}
%     S' = \int_{t_2}^{t_1}{dt\,L\left[x'(t),\frac{d}{dt}x'(t)\right]}
% \end{equation}\label{1.5}
% Note that we don't change the time in integral, this is because time $t$ is a "dummy variable" that can be changed at will in the integral \textit{without affecting action}.
% Importantly, symmetries will always be deformations of the fields (position), not the coordinates (time)\footnote{Please don't mix the transformation of parameters of the integral variable with true symmetries.}.
% So we can change the integral parameter in eq(\hyperref[1.5]{1.5})
% \[S' = \int_{t'_1}^{t'_2}{dt'\,L\left[x'(t'),\frac{dx'(t')}{dt'}\right]}\]
% \[=\int_{t'_1}^{t'_2}{dt\left(1+\frac{d}{dt}\left(\epsilon\frac{dt}{d\epsilon}\right)\right)}\,L\left[x(t)+\frac{d\mathcal{F}}{d\epsilon}\epsilon,\frac{dt}{dt'}\frac{dx'(t')}{dt}\right]\]
% \[=\int_{t'_1}^{t'_2}{dt\left(1+\frac{d}{dt}\left(\epsilon\frac{dt}{d\epsilon}\right)\right)}\,L\left[x(t)+\frac{d\mathcal{F}}{d\epsilon}\epsilon,\left(1-\frac{d}{dt}\left(\epsilon\frac{dt}{d\epsilon}\right)\right)\frac{d\left(x(t)+\frac{d\mathcal{F}}{d\epsilon}\epsilon\right)}{dt}\right]\]
% Then we expand the lagrangian in first order of $\epsilon$,
% \[\int{dt\,\left(1+\frac{d}{dt}\left(\frac{dt}{d\epsilon}\epsilon\right)\right)\left(L+\epsilon\frac{\partial L}{\partial x}\frac{d\mathcal{F}}{d\epsilon}+\frac{\partial L}{\partial \dot{x}}\left(\frac{d}{dt}(\epsilon\frac{d\mathcal{F}}{d\epsilon})-\dot{x}\frac{d}{dt}(\epsilon\frac{dt}{d\epsilon})\right)\right)}\]
% \[=S+\int{dt\,\epsilon\left(\frac{\partial L}{\partial x}\frac{d\mathcal{F}}{\partial\epsilon}+\frac{\partial L}{\partial\dot{x}}\frac{d}{dt}\frac{d\mathcal{F}}{d\epsilon}-\dot{x}\frac{\partial L}{\partial\dot{x}}\frac{d}{dt}\left(\frac{dt}{d\epsilon}\right)+L\frac{d}{dt}\frac{dt}{d\epsilon}\right)}\]
% \[+\int{dt\,\frac{d\epsilon}{dt}}\left(\frac{\partial L}{\partial \dot{x}}\frac{d\mathcal{F}}{d\epsilon}-\dot{x}\frac{\partial L}{\partial \dot{x}}\frac{dt}{d\epsilon} + L\frac{dt}{d\epsilon}\right)\]
% If the infinitesimal transformation is a symmetry, we would have the variation of action $\delta S = S' - S$ vanish, that is, the integration should become 0.
% The second integral, by integration by parts, we would have,
% \begin{equation}
%     \int{dt\,\epsilon\frac{d}{dt}\left(\dot{x}\frac{\partial L}{\partial \dot{x}}\frac{dt}{d\epsilon}-\frac{\partial L}{\partial \dot{x}}\frac{d\mathcal{F}}{d\epsilon} - L\frac{dt}{d\epsilon}\right)} + \text{ boundary term }
% \end{equation}\label{1.6}
% \\\indent The first term can be 0 or a total derivative of time which become boundary term to, together with the boundary term in eq(\hyperref[1.6]{1.6})
% \begin{equation}
%     \left(\frac{\partial L}{\partial x}\frac{d\mathcal{F}}{\partial\epsilon}+\frac{\partial L}{\partial\dot{x}}\frac{d}{dt}\frac{d\mathcal{F}}{d\epsilon}-\dot{x}\frac{\partial L}{\partial\dot{x}}\frac{d}{dt}\left(\frac{dt}{d\epsilon}\right)+L\frac{d}{dt}\frac{dt}{d\epsilon}\right) = 0
% \end{equation}
% \begin{center}
%     Or
% \end{center}
% \[
%     \left(\frac{\partial L}{\partial x}\frac{d\mathcal{F}}{\partial\epsilon}+\frac{\partial L}{\partial\dot{x}}\frac{d}{dt}\frac{d\mathcal{F}}{d\epsilon}-\dot{x}\frac{\partial L}{\partial\dot{x}}\frac{d}{dt}\left(\frac{dt}{d\epsilon}\right)+L\frac{d}{dt}\frac{dt}{d\epsilon}\right) + \text{ boundary term} = \int{dt\,\epsilon\frac{dK}{dt}}
% \]
% Thus we conclude the variation of action would be,
% \begin{equation}
%     \begin{split}
%         \delta S = S' - S\\
%         = \int{dt\,\epsilon\frac{d}{dt}\left(\dot{x}\frac{\partial L}{\partial \dot{x}}\frac{dt}{d\epsilon}-\frac{\partial L}{\partial \dot{x}}\frac{d\mathcal{F}}{d\epsilon} - L\frac{dt}{d\epsilon}+K\right)}
%     \end{split}
% \end{equation}
% Define the conserved charge,
% \begin{equation}
%     Q = \dot{x}\frac{\partial L}{\partial \dot{x}}\frac{dt}{d\epsilon}-\frac{\partial L}{\partial \dot{x}}\frac{d\mathcal{F}}{d\epsilon} - L\frac{dt}{d\epsilon}+K
% \end{equation}
% and we would have,
% \begin{equation}
%     \frac{dQ}{dt} = 0
% \end{equation}


% \subsection{More Friendly Way}
If we would like to discuss the transformation on time, we should use another parameter,
\begin{equation}
    \begin{split}
        t = t(\tau)\\
        q(t(\tau)) = \mathcal{Q}(\tau)
    \end{split}
\end{equation}
Generally, the transformation would be
\begin{equation}
    \begin{split}
        t\rightarrow t' = \tau'(t')\\
        q\rightarrow q'(t') = \mathcal{Q}(\tau')
    \end{split}
\end{equation}
and the variation of position and time\footnote{The parameter should be the same.},
\begin{equation}
    \begin{split}
        \delta_{\text{S}}q = q'(t)-q(t) =\mathcal{Q}'(\tau) - \mathcal{Q}(\tau)= \delta_{\text{S}}\mathcal{Q}\\
        \delta_{\text{S}}t = t'-t = \tau'(t) -\tau(t) = \delta_{\text{S}}\tau
    \end{split}
\end{equation}
Rewrite the original action,
\begin{equation}
    S = \int_{t_1}^{t_2}{dt\,L[q(t), \dot{q}(t)]} = \int_{\tau_1}^{\tau_2}{d\tau\,\left(\frac{dt}{d\tau}\right)L[\mathcal{Q}, \frac{d}{d\tau}\mathcal{Q}, t]}
\end{equation}
Redefine the new lagrangain,
\begin{equation}
    \mathbb{L}\left[\mathcal{Q}, \frac{d}{d\tau}\mathcal{Q},t,\frac{dt}{d\tau}\right] = \frac{dt}{d\tau}L
\end{equation}
Now consider the infinitesimal symmetry transformation, and the new action would become,
\begin{equation}
    S' = \int{d\tau\,\mathbb{L}\left[\mathcal{Q}', \frac{d\mathcal{Q}}{d\tau}', t',\frac{dt}{d\tau}'\right]} + \int{d\tau\frac{dK}{d\tau}}
\end{equation}
Adding a term of total derivative for general consideration, and we know it wouldn't affect the equation of motion.
Therefore we can do the variation of action and lagrangian,
\begin{equation}
    \delta S = S' - S = 0
\end{equation}
The variation of action should vanish because we assume it is symmetry infinitesimal transformation.

\end{document}